\documentclass[a4paper, 12pt]{article}

\usepackage[T1]{fontenc}
\usepackage[polish]{babel} 
\usepackage[utf8]{inputenc} 
\usepackage{indentfirst}
\usepackage{setspace}
\usepackage{fancyhdr}
\usepackage{hyperref}
\usepackage{pdfpages}
\usepackage{listings}
\usepackage{color}
\usepackage{graphicx}
\usepackage{enumitem}
\usepackage{latexsym}
\pagestyle{fancy} 
\hypersetup{
    colorlinks=true,
    linkcolor=blue,
    filecolor=magenta,      
    urlcolor=cyan,
}
\pagestyle{fancy} 
\pagestyle{fancy} 
\newcommand{\mainmatter}{\clearpage \cfoot{\thepage\ of \pageref{LastPage}}
\pagenumbering{arabic}}

\begin{document}
	\begin{titlepage}
		
		\begin{center}
			\Large Politechnika Warszwska\\ 
			\Large Wydział: \large Elektryczny\\
			\Large Kierunek: \large Informatyka Stosowana\\
			\Large Specjalność:	 \large Inżynieria Oprogramowania\\
    	\vspace{5cm}
    		\Large\textit{\textbf{Sprawozdanie  
    		\\Aplikacja WWW z wykorzystaniem REACT oraz GO-lang}}\\ 
		\vspace{3cm}
		\end{center} 

		\hfill\begin{minipage}{0.65\textwidth}
			\Large Wykonali:\newline
				1. Agratina Oleksandr, 287571\newline
				2. Manevych Andrii, 294867\newline
				3. Prakapets Ivan, 295139\newline
				4. Starastsenka Uladzislau, 294534
		\vspace{\baselineskip}
		\end{minipage}
		
		
		\hfill\begin{minipage}{0.65\textwidth}
			\Large Opiekun naukowy:\newline
		 		dr inż. Jarosław Wilk
		\end{minipage}
		
		\hfill\begin{minipage}{0.5\textwidth}
		\vspace{1cm}
			\Large Data: 16.05.2020
			\vspace{\baselineskip}
		\end{minipage}

	\end{titlepage}
\newpage
\mainmatter
\setlength{\headheight}{15pt}
\doublespacing
\tableofcontents
\newpage

	\section{Opis Ogólny}
		\subsection{Nazwa programu} 
			\textbf{Nazwa programu:} Wyszukiwanie przepisów wobiec produktów Generator przepisów.\newline
			%gra$\_$w$\_$zycie.}
			
		\subsection{Poruszany problem}
			\hspace*{1cm} W czasy teraźniejsze potrzebujemy aplikacji, która będzie umożliwiać znaleźć przepis na podstawie wybranych prodóktów, które użytkownik posiada. 
			Ta aplikacja będzie przydatna dla wielu studetów ( które nie mają za dużo  prodóktów dla przygotowywania jedzenia ). Analogicznie dla dużej ilości ludzi.
        \subsection{Motto} Usługa wyszukiwania receptur według składników pomoże ci!
\newpage

\section{Praca w zespole}
\subsection{Narzędzie komunikacji w zespole}
\hspace*{1cm} Wykorzystaliśmy takie narzędzie jak GitHub, Telegram, Word Online.\newline
\hspace*{1cm} Iteracji oraz zadania były rozpisane w narzędziu Trello. Bardzo ono nam pomogło.
Były zaznaczone ludzie do realizacji zadań oraz terminy wykonania zadań. Oczywiście, zdarzały się takie momenty kiedy, terminy były przekraczane, i te zadania były przeniesione do kolejnej iteracji realizacji projektu.
\subsubsection{Napotkane problemy w komunikacji}
\subsubsection{Napotkane problemy w zespole}
\subsubsection{Rozwiązanie problemów}
\subsection{Wnioski}
\section{Aplikacja}
Tą cześć narazie zostawiam, bo nie mamy aplikacji.
\subsection{Narzędzie}
\subsection{Technologia}
\subsection{Funkcjonalność}
\label{LastPage}~
\label{LastPageOfBackMatter}~		
\end{document}