\documentclass[a4paper, 12pt]{article}

\usepackage[T1]{fontenc}
\usepackage[polish]{babel} 
\usepackage[utf8]{inputenc} 
\usepackage{indentfirst}
\usepackage{setspace}
\usepackage{fancyhdr}
\usepackage{hyperref}
\usepackage{pdfpages}
\usepackage{listings}
\usepackage{color}
\usepackage{graphicx}
\usepackage{enumitem}
\usepackage{latexsym}
\pagestyle{fancy} 
\hypersetup{
    colorlinks=true,
    linkcolor=blue,
    filecolor=magenta,      
    urlcolor=cyan,
}
\pagestyle{fancy} 
\pagestyle{fancy} 
\newcommand{\mainmatter}{\clearpage \cfoot{\thepage\ of \pageref{LastPage}}
\pagenumbering{arabic}}

\begin{document}
	\begin{titlepage}
		
		\begin{center}
			\Large Politechnika Warszwska\\ 
			\Large Wydział: \large Elektryczny\\
			\Large Kierunek: \large Informatyka Stosowana\\
			\Large Specjalność:	 \large Inżynieria Oprogramowania\\
    	\vspace{5cm}
    		\Large\textit{\textbf{SPECYFIKACJA FUNKCJONALNA 
    		\\Aplikacja WWW z wykorzystaniem REACT oraz GO-lang}}\\ 
		\vspace{3cm}
		\end{center} 

		\hfill\begin{minipage}{0.65\textwidth}
			\Large Wykonali:\newline
				1. Agratina Oleksandr, 287571\newline
				2. Manevych Andrii, 294867\newline
				3. Prakapets Ivan, 295139\newline
				4. Starastsenka Uladzislau, 294534
		\vspace{\baselineskip}
		\end{minipage}
		
		
		\hfill\begin{minipage}{0.65\textwidth}
			\Large Opiekun naukowy:\newline
		 		dr inż. Jarosław Wilk
		\end{minipage}
		
		\hfill\begin{minipage}{0.5\textwidth}
		\vspace{1cm}
			\Large Data: 20.03.2020
			\vspace{\baselineskip}
		\end{minipage}

	\end{titlepage}
\newpage
\mainmatter
\setlength{\headheight}{15pt}
\doublespacing
\tableofcontents
\newpage

	\section{Opis Ogólny}
		\subsection{Nazwa programu} 
			\textbf{Nazwa programu:} \texttt{wyszukiwanie receptów wobiec produktów, które masz w obecnej chwili} LUB Generator receptur.\newline
			PODOBNA APLIKACJA DEMO VERSION: \href{http://recepty-po-ingredientam.ru/}{DEMO}\newline
			PODOBNA APLIKACJA FULL: \href{https://mnevkusno.ru/}{FULL}
			%gra$\_$w$\_$zycie.}
			
		\subsection{Poruszany problem}
			\hspace*{1cm} W czasy teraźniejsze potrzebujemy aplikacji, która będzie umożliwiać znaleźć recept na podstawie wybranych prodóktów, które posiadasz. 
			Ta aplikacja będzie przydatna dla wielu studetów ( które nie mają za dużo  prodóktów dla przygotowywania jedzenia ). Analogicznie dla dużej ilości ludzi, bo w czasy dzisiejsze większość restauracji jest zamknięta. (KORONAWIRUS:)) 
			
Usługa wyszukiwania receptur według składników pomoże ci!
\newpage
	\section{Opis funkcjonalności}
		\subsection{Możliwości programu}
			\hspace*{1cm} Program będzie zawierał następujące możliwości:
		
 		\begin{enumerate}
 		    \item Pasek dla wyszukiwania receptur(wpisywania produtów).
 		    \item Było by fajnie, żeby kiedy wpisujesz na przykład mil...(podpowiada z listy znak po znaku)
 		    \item Lista receptur.
 			\item Użytkownik ma możliwość wyboru kategorij 
 			\item Po wyszukiwaniu receptur sortowanie według opinij innych użytkowników?????
 			\item Sprawdzanie poprawności podanych argumentów, danych.
 			\item Obsługa różnych błędnych danych.
 			\item Ulubione receptury ( nie wiem czy potrzebne) 
 			\item Użytkownik ma możliwość wyboru kuchni(arabska,polska???? nie czy potrzebne).
  			\item Zapisanie stanów wyboru(historia) pewnego użytkownika ( po zalogowaniu(FULL VERSION) DOPOKI NIE PATRZEC)
 		\end{enumerate}
 		\subsection{Filtr wyszukiwania}
 		\subsubsection{Kategoria receptur}
 		\begin{enumerate}
 		    \item Filt dotyczące wyboru poprzez użytkownika kategorij receptur
 		    \begin{itemize}
 		        \item śniadanie
 		        \item Danie główne (drugie)
 		        \item napój
 		        \item sałatka
 		        \item desery
 		        \item zupa
 		        \item inne
 		    \end{itemize}
 		\end{enumerate}
 		\subsection{Co zawiera receptura?}
 		Recept zawiera następujące dane: 
 		\begin{itemize}
 		    \item Nazwa receptu
 		    \item Ingridients (product) 
 		    \item Przykładowy czas przygotowywania
 		    \item Krok po kroku, co trzeba robić, żeby ugotować danie (Sposób przygotowania)
 		    \item ID RECEPTU
 		    \item Opinij , zwiazdy, ranking. 
 		\end{itemize}
 		\subsection{Jak korzystać się z APLIKACJI?}
 			\hspace*{1cm} Program ma interfejs graficzny. Z tego powodu korzystanie jest łatwe. Co musisz zrobić, żeby korzystać z aplikacji? \newline
            Kroki:
            \begin{enumerate}
                \item Wejdź do strony HTTP/HTTPS
                \item Wpisz produkty, które posiadasz  do paski wyszukiwania
                \item !!!!OPCJONALNIE!!! wybierz kategorii (PO DEFAULTU bedze wszystkie kategorii)
                \item Nacisknij przycisk znajdz
                \item Będzie pokazana lista receptur, które posiadają wpisane produkty w punkcie 2
                \item Wybierz recept, którym zainteresowałeś
                \item Przejście do następnej strony z jednym receptem
                \item Wykonywaj polecenia w recepcie
                \item SMACZNEGO!!!!
                \item Wyjdz ze strony lub ponów kroki 2-8
            \end{enumerate}
\newpage
	\section{Format danych i struktura katalogów} 
		\subsection{Struktura katalogów}
			\hspace*{1cm} Aplikacja będzie zawierała kilka katalogów.
			Katalog głowny GeneracjaReceptur.
			Podkatalog backend będzie zawierał pliki źródłow. Podkatalog \texttt{''frondend''} będzie zawierał pliki \texttt{*.css} oraz \texttt{*.html} oraz .js. NIE JESTEM PEWNY 
		
		\subsection{Dane wejściowe}
			\hspace*{1cm} Aplikacja WWW otrzymuje dane wejściowe (wpisane produkty) poprzez użytkownika. 
		
		\subsection{Dane wyjściowe}
			\hspace*{1cm} W wyniku wpisania produktó na WWW stronie, zostaną  wygenerowane receptury dań zawieracjące opis, nazwa, czas i tak dalej.
\newpage
	\section{Obsługa sytuacji błędnych}
		\hspace{1cm} Program będzie obsługiwał błędne dane z odpowiednimi komunikatami:
		\renewcommand{\labelitemi}{$\ast$}
		
		\begin{itemize}
			\item \textit{Niepoprawny wpisany produkt. Spróbuj ponownie}
 			\item \textit{Nie wybrana żadna kategoria dań}
 			\item \textit{System nie znalazł receptur, które zawierają wpisane produkty. Spróbuj wpisać mniej}
 			\item \textit{Cos jescze chyba mozna dopisac}
 			\item \textit{Nie wolno wpisywac cyfr, TYLKO Litery}
		\end{itemize}
	\section{Testowanie}
		\hspace{1cm} DEMO VERSION 
\label{LastPage}~
\label{LastPageOfBackMatter}~		
\end{document}